\documentclass[11pt,largemargins]{homework}

\newcommand{\hwname}{------ ------}
\newcommand{\hwemail}{-}
\newcommand{\hwtype}{Chapter}
\newcommand{\hwnum}{1}
\newcommand{\hwclass}{Linear Algebra}
\newcommand{\hwlecture}{0}
\newcommand{\hwsection}{Z}

\begin{document}
\maketitle

\textbf{\large{End of Chapter Exercises}}

\hfill 

\textbf{\large{Section 1.1} Introduction}
%%%% ONE %%%%
\question
Determine whether the vectors emanating from the origin and terminating at the following pairs of points are parallel.
\begin{alphaparts}
    \questionpart
    (3,1,2) and (6,4,2)

    \questionpart
    (-3,1,7) and (9,-3,-21)

    \questionpart
    (5,-6,7) and (-5,6,-7)

    \questionpart
    (2,0,-5) and (5,0,-2)

\end{alphaparts}

%%%% TWO %%%%
\question
Find the equations of the lines through the following pairs of points in space.
\begin{alphaparts}
    \questionpart
    (3,-2,4) and (-5,7,1)

    \questionpart
    (2,4,0) and (-3,-6,0)

    \questionpart
    (3,7,2) and (3,7,-8)

    \questionpart
    (-2,-1,5) and (3,9,7)

\end{alphaparts}

%%%% THREE %%%%
\question
Find the equations of the planes containing the following points in space.
\begin{alphaparts}
    \questionpart
    (2,-5,-1), (0,4,6) and (-3,7,1)

    \questionpart
    (3,-6,7), (-2,0,-4) and (5,-9,-2)

    \questionpart
    (-8,2,0), (1,3,0) and (6,-5,0)

    \questionpart
    (1,1,1), (5,5,5) and (-6,4,2)

\end{alphaparts}

%%%% FOUR %%%%
\question
What are the coordinates of the vector $0$ in the Eucliean plane that satisfies Property 3? (There exists a vector 
denoted $0$ such that $x+0=x$ for each vector $x$.)

%%%% FIVE %%%%
\question
Prove that if the vector $x$ emanates from the origin of the Euclidean plane and terminates at the point with coordinates 
$(a_1,a_2)$, then the vector $tx$ that emanates from the origin terminates at the point with coordinates $(ta_1,ta_2)$.

%%%% SIX %%%%
\question
Show that the midpoint of the line segment joining the points $(a,b)$ and $(c,d)$ is $((a+c)/2, (b+d)/2)$.

%%%% SEVEN %%%%
\question
Prove that the diagonals of a parallelogram bisect each other.


\hfill 

\textbf{\large{Section 1.2} Vector Spaces}
\setcounter{questionCounter}{0}
%%%% ONE %%%%
\question

%%%% TWO %%%%
\question

%%%% THREE %%%%
\question

%%%% FOUR %%%%
\question

%%%% FIVE %%%%
\question


\textbf{\large{Section 1.3} Subspaces}
\setcounter{questionCounter}{0}
%%%% ONE %%%%
\question

%%%% TWO %%%%
\question

%%%% THREE %%%%
\question

%%%% FOUR %%%%
\question

%%%% FIVE %%%%
\question


\hfill 

\textbf{\large{Section 1.4} Linear Combinations and Systems of Linear Equations}
\setcounter{questionCounter}{0}
%%%% ONE %%%%
\question

%%%% TWO %%%%
\question

%%%% THREE %%%%
\question

%%%% FOUR %%%%
\question

%%%% FIVE %%%%
\question


\hfill 

\textbf{\large{Section 1.5} Linear Dependence and Linear Independence}
\setcounter{questionCounter}{0}
%%%% ONE %%%%
\question

%%%% TWO %%%%
\question

%%%% THREE %%%%
\question

%%%% FOUR %%%%
\question

%%%% FIVE %%%%
\question


\hfill 

\textbf{\large{Section 1.6} Bases and Dimension}
\setcounter{questionCounter}{0}
%%%% ONE %%%%
\question

%%%% TWO %%%%
\question

%%%% THREE %%%%
\question

%%%% FOUR %%%%
\question

%%%% FIVE %%%%
\question


\hfill 

\textbf{\large{Section 1.7} Maximal Linearly Independent Subsets}
\setcounter{questionCounter}{0}
%%%% ONE %%%%
\question

%%%% TWO %%%%
\question

%%%% THREE %%%%
\question

%%%% FOUR %%%%
\question

%%%% FIVE %%%%
\question

%%%% SIX %%%%
\question

%%%% SEVEN %%%%
\question

%%%% EIGHT %%%%
\question

%%%% NINE %%%%
\question

%%%% TEN %%%%
\question







\end{document}









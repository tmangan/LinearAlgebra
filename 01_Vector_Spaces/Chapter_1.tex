\documentclass[11pt,largemargins]{homework}

\newcommand{\hwname}{------ ------}
\newcommand{\hwemail}{-}
\newcommand{\hwtype}{Chapter}
\newcommand{\hwnum}{1}
\newcommand{\hwclass}{Linear Algebra}
\newcommand{\hwlecture}{0}
\newcommand{\hwsection}{Z}

\begin{document}
\maketitle

\textbf{\large{End of Chapter Exercises}}

\hfill 

\textbf{\large{Section 1.1} Introduction}
%%%% ONE %%%%
\question
Determine whether the vectors emanating from the origin and terminating at the following pairs of points are parallel.
\begin{alphaparts}
    \questionpart
    (3,1,2) and (6,4,2)

    \questionpart
    (-3,1,7) and (9,-3,-21)

    \questionpart
    (5,-6,7) and (-5,6,-7)

    \questionpart
    (2,0,-5) and (5,0,-2)

\end{alphaparts}

%%%% TWO %%%%
\question
Find the equations of the lines through the following pairs of points in space.
\begin{alphaparts}
    \questionpart
    (3,-2,4) and (-5,7,1)

    \questionpart
    (2,4,0) and (-3,-6,0)

    \questionpart
    (3,7,2) and (3,7,-8)

    \questionpart
    (-2,-1,5) and (3,9,7)

\end{alphaparts}

%%%% THREE %%%%
\question
Find the equations of the planes containing the following points in space.
\begin{alphaparts}
    \questionpart
    (2,-5,-1), (0,4,6) and (-3,7,1)

    \questionpart
    (3,-6,7), (-2,0,-4) and (5,-9,-2)

    \questionpart
    (-8,2,0), (1,3,0) and (6,-5,0)

    \questionpart
    (1,1,1), (5,5,5) and (-6,4,2)

\end{alphaparts}

%%%% FOUR %%%%
\question
What are the coordinates of the vector $0$ in the Eucliean plane that satisfies Property 3? (There exists a vector 
denoted $0$ such that $x+0=x$ for each vector $x$.)

%%%% FIVE %%%%
\question
Prove that if the vector $x$ emanates from the origin of the Euclidean plane and terminates at the point with coordinates 
$(a_1,a_2)$, then the vector $tx$ that emanates from the origin terminates at the point with coordinates $(ta_1,ta_2)$.

%%%% SIX %%%%
\question
Show that the midpoint of the line segment joining the points $(a,b)$ and $(c,d)$ is $((a+c)/2, (b+d)/2)$.

%%%% SEVEN %%%%
\question
Prove that the diagonals of a parallelogram bisect each other.


\hfill 

\textbf{\large{Section 1.2} Vector Spaces}
\setcounter{questionCounter}{0}
%%%% ONE %%%%
\question
Label the following statements as True or False.
\begin{alphaparts}
    \questionpart
    Every vector space contains a zero vector.
    \questionpart
    A vector space may have more than one zero vectos.
    \questionpart
    In any vector space, $ax=bx$ implies that $a=b$.
    \questionpart
    In any vector space, $ax=ay$ implies that $x=y$.
    \questionpart
    A vector in $\mathbb{F}^n$ may be regarded as a matrix in $M_{n\times 1}(F)$.
    \questionpart
    An $m\times n$ matrix has $m$ columns and $n$ rows.
    \questionpart
    In $\mathbf{P}(F)$, only polynomials of the same degree may be added.
    \questionpart
    If $f$ and $g$ are polynomicals of degree $n$, then $f+g$ is a polynomial of degree $n$.
    \questionpart
    If $f$ is a polynomial of degree $n$ and $c$ is a nonzero scalar, then $cf$ is a polynomial of degree $n$.
    \questionpart
    A nonzero scalar of $F$ may be considered to be a polynomial in $\mathbf{P}(F)$ having degree zero.
    \questionpart
    Two functions in $\mathcal{F}(S,F)$ are equal if and only if they have the same value at each element of $S$.
\end{alphaparts}

%%%% TWO %%%%
\question
Write the zero vector of $M_{3\times 4}(F)$.

%%%% THREE %%%%
\question
If 
\[
M=
\begin{pmatrix}
    1 & 2 & 3 \\
    4 & 5 & 6
\end{pmatrix}
\]

what are $M_{1,3}$, $M_{2,1}$, and $M_{2,2}$?

%%%% FOUR %%%%
\question
Perform the indicated operations.
\begin{alphaparts}
    \questionpart
        $\begin{pmatrix}
            2 & 5 & -3 \\
            1 & 0 & 7 
        \end{pmatrix}
        +
        \begin{pmatrix}
            4 & -2 & 5 \\
            -5 & 3 & 2 
        \end{pmatrix}$

    \questionpart
        $\begin{pmatrix}
            -6 & 4  \\
            3 & -2  \\ 
            1 & 8
        \end{pmatrix}
        +
        \begin{pmatrix}
            7 & -5  \\
            0 & -3  \\
            2 & 0
        \end{pmatrix}$

    \questionpart
        $4 
        \begin{pmatrix}
            2 & 5 & -3 \\
            1 & 0 & 7 
        \end{pmatrix}$

    \questionpart
        $-5
        \begin{pmatrix}
            -6 & 4  \\
            3 & -2  \\
            1 & 8
        \end{pmatrix}$

    \questionpart
    $(2x^4-7x^3+4x+3)+(8x^3+2x^2-6x+7)$
    \questionpart
    $(-3x^3+7x^2+8x-6)+(2x^3-8x+10)$
    \questionpart
    $5(2x^7-6x^4+8x^2-3x)$
    \questionpart
    $3(x^5-2x^3+4x+2)$

\end{alphaparts}

%%%% FIVE %%%%
\question

%%%% SIX %%%%
\question

%%%% SEVEN %%%%
\question
Let $S=\{0,1\}$ and $\mathbb{F}=\mathbb{R}$. In $\mathcal{F}(S,R)$, show that $f=g$ and $f+g=h$, 
where $f(t)=2t+1$, $g(t)=1+4t-2t^2$, and $h(t)=5^t+1$.

%%%% EIGHT %%%%
\question
In any vector space $\mathbf{V}$, show that $(a+b)(x+y)=ax+ay+bx+by$ for any $x,y \in \mathbf{V}$ and any $a,b \in \mathbb{F}$.

%%%% NINE %%%%
\question
Prove Corollaries 1 and 2 of Theorem 1.1 and Theorem 1.2(c).

%%%% TEN %%%%
\question
Let $\mathbf{V}$ denote the set of all differentiable real-valued functions defined on the real line. Prove that $\mathbf{V}$ is 
a vector space with the operations of addition and scalar multiplication defined in Example 3.

%%%% ELEVEN %%%%
\question
Let $\mathbf{V}=\{\mathbf{0}\}$ consist of a single vector \textbf{0} and define 
$\mathbf{0}+\mathbf{0}=\mathbf{0}$ and $c\mathbf{0}=\mathbf{0}$ for each scalar $c$ in $\mathbb{F}$. 
Prove that \textbf{V} is a vector space over $\mathbb{F}$. ($\mathbf{V}$ is called the \textbf{zero vector space}.)

%%%% TWELVE %%%%
\question
A real-valued function $f$ defined on the real line is called an \textbf{even function} if $f(-t)=f(t)$ for each real number $t$. 
Prove that the set of even functions defined on the real line with the operations of addition and scalar multiplication 
defined in Example 3 is a vector space.

%%%% THIRTEEN %%%%
\question
Let \textbf{V} denote the set of ordered pairs of real numbers. If $(a_1,a_2)$ and $(b_1,b_2)$ are elements of \textbf{V} and 
$c\in \mathbb{R}$, define 
$$(a_1,a_2)+(b_1,b_2)=(a_1+b_1,a_2b_2)\text{ and } c(a_1,a_2)=(ca_1,a_2)$$

Is \textbf{V} a vector space over $\mathbb{R}$ with these operations? Justify your answer.

%%%% FOURTEEN %%%%
\question
Let $\mathbf{V}=\{(a_1,a_2,...,a_n)\;|\;a_i\in \mathbb{C} \text{ for } i=1,2,...,n\}$; 
so \textbf{V} is a vector space over $\mathbb{R}$ by Example 1.
If \textbf{V} a vector space over the field of complex numbers with the operations of coordinatewise addition and multiplication?

%%%% FIFTEEN %%%%
\question
Let $\mathbf{V}=\{(a_1,a_2,...,a_n)\;|\;a_i\in \mathbb{R} \text{ for } i=1,2,...,n\}$; 
so \textbf{V} is a vector space over $\mathbb{R}$ by Example 1.
Is \textbf{V} a vecotr space over the field of complex numbers with the operations of coordinatewise addition and multiplication?

%%%% SIXTEEN %%%%
\question
Let \textbf{V} denote the set of all $m\times n$ matrices with real entries; so \textbf{V} is a vector space over $\mathbb{R}$ by Example 2.
Let $\mathbb{Q}$ be the field of rational numbers. Is \textbf{V} a vector space over $\mathbb{F}$ with the usual definitions of matrix 
addition and scalar multiplication?

%%%% SEVENTEEN %%%%
\question
Let $\mathbf{V}=\{(a_1,a_2) \;|\; a_1,a_2 \in \mathbb{F}\}$, where $\mathbb{F}$ is a field. 
Define addition of elements of \textbf{V} coordinatewise 
and for $c \in \mathbb{F}$ and $(a_1,a_2)\in \mathbf{V}$, define 
$$c(a_1,a_2)=(a_1,0).$$
Is \textbf{V} a vector space over $\mathbb{F}$ with these operations? Justify your answer.

%%%% EIGHTEEN %%%%
\question
Let $\mathbf{V}=\{(a_1,a_2) \;|\; a_1,a_2 \in \mathbb{R}\}$. For $(a_1,a_2),(b_1,b_2) \in \mathbf{V}$ and $c\in \mathbb{R}$,
define 
$$(a_1,a_2)+(b_1,b_2)=(a_1+2b_1,a_2+3b_2)\text{ and }c(a_1,a_2)=(ca_1,ca_2).$$
Is \textbf{V} a vector space over $\mathbb{R}$ with these operations?

%%%% NINETEEN %%%%
\question
Let $\mathbf{V}=\{(a_1,a_2) \;|\; a_1,a_2 \in \mathbb{R}\}$. Define addition of elements of \textbf{V} coordinatewise, 
and for $(a_1,a_2)$ in \textbf{V} and $c \in \mathbb{R}$, define
\[ 
    c(a_1,a_2)=
\begin{cases}
    (0,0)    & \text{if }c=0\\
    \left(ca_1,\frac{a_2}{c}\right) & \text{if }c\neq 0
\end{cases}
\]
Is \textbf{V} a vector space over $\mathbb{R}$ with these operations?

%%%% TWENTY %%%%
\question
Let \textbf{V} be the set of sequences $\{a_n\}$ of real numbers. For $\{a_n\},\{b_n\}\in \mathbf{V}$ and any real number $t$, define
$$\{a_n\}+\{b_n\}=\{a_n+b+n\}\text{ and }t\{a_n\}=\{ta_n\}$$
Prove that, with these operations, \textbf{V} is a vector space over $\mathbb{R}$.

%%%% TWENTY ONE %%%%
\question
Let \textbf{V} and \textbf{W} be vector spaces over a field $\mathbb{F}$. Let 
$$\mathbf{Z}=\{(v,w) \;|\; v\in \mathbf{V}\text{ and }w\in\mathbf{W} \}.$$
Prove that \textbf{Z} is a vector space over $\mathbb{F}$ with the operations 
$$(v_1,w_1)+(v_2,w_2)=(v_1+v_2,w_1+w_2)\text{ and }c(v_1,w_1)=(cv_1,cw_1).$$

%%%% TWENTY TWO %%%%
\question
How many matrices are there in the vector space $M_{m\times n}(\mathbb{Z}_2)$?



\hfill

\textbf{\large{Section 1.3} Subspaces}
\setcounter{questionCounter}{0}
%%%% ONE %%%%
\question

%%%% TWO %%%%
\question

%%%% THREE %%%%
\question

%%%% FOUR %%%%
\question

%%%% FIVE %%%%
\question


\hfill 

\textbf{\large{Section 1.4} Linear Combinations and Systems of Linear Equations}
\setcounter{questionCounter}{0}
%%%% ONE %%%%
\question

%%%% TWO %%%%
\question

%%%% THREE %%%%
\question

%%%% FOUR %%%%
\question

%%%% FIVE %%%%
\question


\hfill 

\textbf{\large{Section 1.5} Linear Dependence and Linear Independence}
\setcounter{questionCounter}{0}
%%%% ONE %%%%
\question

%%%% TWO %%%%
\question

%%%% THREE %%%%
\question

%%%% FOUR %%%%
\question

%%%% FIVE %%%%
\question


\hfill 

\textbf{\large{Section 1.6} Bases and Dimension}
\setcounter{questionCounter}{0}
%%%% ONE %%%%
\question

%%%% TWO %%%%
\question

%%%% THREE %%%%
\question

%%%% FOUR %%%%
\question

%%%% FIVE %%%%
\question


\hfill 

\textbf{\large{Section 1.7} Maximal Linearly Independent Subsets}
\setcounter{questionCounter}{0}
%%%% ONE %%%%
\question

%%%% TWO %%%%
\question

%%%% THREE %%%%
\question

%%%% FOUR %%%%
\question

%%%% FIVE %%%%
\question

%%%% SIX %%%%
\question

%%%% SEVEN %%%%
\question

%%%% EIGHT %%%%
\question

%%%% NINE %%%%
\question

%%%% TEN %%%%
\question







\end{document}








